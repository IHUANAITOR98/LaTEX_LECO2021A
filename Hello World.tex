\documentclass{article}

%vamos a exportar
\usepackage{amsmath}  %matrices
\usepackage{graphicx} %imagenes

%precargar ruta de imagenes
\graphicspath{ {Imagenes/} }

\title{Hello WORLD!!}
\author{Said García}
\date{29 de Marzo 2022}

\begin{document}

  \begin{titlepage}
    \maketitle
    \begin{abstract}
      Este es un documento de pruebas que servirá \par
      a manera de playground
    \end{abstract}
    \begin{center}
      Este texto se supone aparecerá en el centro
      
      Con la relación $x=y$  \par
      %aqui se inserta la imagen
      \includegraphics{cide.jpg}
    \end{center}

  \end{titlepage}

  
  \begingroup
    \begin{section}*{Esta va a ser la de puro texto}
      \paragraph*{Así es como se inicia el párrafo}
      Lorem Ipsum is simply dummy text of the printing and typesetting 
      industry. Lorem Ipsum has been the industry's standard dummy text 
      ever since the 1500s, when an unknown printer took a galley of type
       and scrambled it to make a type specimen book. It has survived 
       not only five centuries, but also the leap into electronic 
       typesetting, remaining essentially unchanged. It was popularised 
       in the 1960s with the release of Letraset sheets containing Lorem 
       Ipsum passages, and more recently with desktop publishing software 
      like Aldus PageMaker including versions of Lorem Ipsum. 
      \par
  
      %así se escribe una cita
      \begin{quote}
        No puede comprenderse nada del saber económico si no se sabe cómo se 
        ejercía, en su cotidianeidad, el poder, y el poder económico. - Michel 
        Foucalt.
        \par 
      \end{quote}

      
      \paragraph*{Este es un párrafo alineado a la izquierda}

  \endgroup

  %así se da un salto de página
  \newpage

  \begingroup
  %título de una sección
    \section*{Formas de iniciar el modo maténamtico}
      %con \[\] es para ecuaciones/modelos cortos (preferiblemente)
      \[ TMS = \frac{
          \frac{
            \partial U}
          {
            \partial X_1}
        }
        {
          \frac{
            \partial U}
          {
            \partial X_2
          }
        }
      \]

      %Para hacer incersiones dentro de un bloque de texto se usa \(\)
      \paragraph[¿Aquí que es?]{Vamos a iniciar un párrafo otra vez}
      El cociente de precios se expresa: \(\frac{P_1}{P_2}\) \par
      
      %para una secuencia de varias entradas metemáticas usamos:
      \begin{math}
        \int_a^b X^2 - (Y-a)^2  \\
        \int^a_b X^2 - ( Y^2 -2Y 4 )
        \\
        \\
        \sqrt{x_1^3}
        \\
        \\
      \end{math}
      %Para una matriz, es necesario exportar el packete "amsmath"
      \begin{math}
        \begin{bmatrix}
          2 & 5 & 9 \\
          3 & 6 & 0 \\
          4 & 7 & 1
        \end{bmatrix}
      \end{math}
  \endgroup


\end{document}